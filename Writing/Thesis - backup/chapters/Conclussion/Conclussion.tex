Even though it is often hard to grasp, the interactions between individual elements and their collective emergent behavior are defining for many natural systems. In drug discovery, understanding the interplay between the microscopic and the macroscopic is vital; changes occur on minute intracellular dynamics -- with large-scale effects on organ, human, or even population scale.\\ 
% \footnote{The Danish population consist of $10^{20}$ cells}

Through examining the physics of the cell, biology of gastrulation, and computer science of pre-existing in silico solutions, we have proposed a simple and biologically sensible computer model. Using data of the genetic expression in the fruit fly embryo, the gastrulation was condensed to depend on only a handful of parameters on single cell level. \\

Using a novel approach for the field, our off-grid agent-based model allows for a full-embryo simulation of the \textit{Drosophila} morphogenesis. This leads to a range of new possible analyses which have never been performed before: An in silico recreation of the much-studied posterior midgut invagination, and an examination of the interaction between both active and reactive domains of the embryo. \\

The solution was validated by large-scale visual analyses of the temporally evolving morphology. By identifying a number of key dynamic events in the embryogenesis, the timeline was compared to known in vivo phenomena. We have found general qualitative agreement, albeit with some obvious shortcomings, mainly the absence of anisotropy in cellular shape.\\

After visual verification of the model, a number of numerical metrics were devised to quantify the agreement between simulation and reality. These metrics grant us quantitative evidence of our model's ability to correctly recuperate the gastrulation. Finally, we introduced known and unknown mutants, observed the model response, and analyzed the ramifications of these perturbations. \\\\

Turing showed that simple interplay between chemicals can create arbitrarily complex biopatterning. We have taken a step toward demonstrating how minimal, axiomatic environmental information, and straightforward, naturally feasible single-cell rules can lead to the creation of physical form.\\


% If you are studying morphogenesis, you are certain to encounter the gastrulation of the fruit fly. As people have studied the fly troughout the years, a self-perpetuating effect has brought it to be the de-facto model organism for much of the field. \\

% While looking at the [small parts] in many cases can lead to crudely reductionist conclusions, it is sometimes the only obvious.

% We have seen that even if many biological systems are "more than than the sum of its parts", having a model for the parts can allow for intere.

% We have shown that complex time-[something] of events can arise from the physically simple interactions with boundary conditions and mechanical feedback-loops.


% We wanted to probe the feasibility of [recuperating] using a point-based simulation with minimal rule-set. We have kept every interaction as biologically defensible as possible, keeping in mind that most coordinated efforts in nature are emergent -- with all(/most?) large-scale information sharing contained in the initial conditions.
% After [trusting] the phenomenological validity of our simulation, we . Checking its stability under perturbations in parameters and [YYY].\\

% For the qualitative and quantitative comparisons between silico and vivo we found the following.

% Folds on the top (dorsal) side would be a welcome addition. The literature agrees \reph\todo{Link to the paper where it show it happens under stress}. Cells shape change for added motion on the bottom. We saw, when trying to have more horizontal pressure come from the \vf{ventral furrow}. Active intercalations (as described in Section \ref{sec:germ-band}) are simply not enough. This, we found out, was also corroborated by the literature \cite{detder}.

% Explicit timing does not seem needed!

% Using our novel modeling approach we have found the interplay between seemingly disparate biomechanical events to be [vital]. This is   unsurprisingly as the different   

% The fractal-like YYY of emergence across multiple [phsyical scales] is defining for natural phenomena.

% Turing


% \textbf{We are in a way the first who can simulate the posterior invagination (PMG) \pmg{A3} as this requires both successful modelling of \vf{A1} and \gb{A2}. Each of which has had multiple YYY before, but never in a way where combining them was possible}


% \section{Questions}
% \todo{Give me all the questions again}
% Q0 - Mimimale\\
% Q1 - Are BC/IC sufficient?\\
% Q2 - Given a bottom-up, full scale model, what can we say re: domain-interactions\\
% \\\\
% Result 1: \
% Result 2a: Comparison between Silico mutants and Vivo mutants\\
% Result 2a: Comparison between Silico mutants and Silico optimal (remember to include PCP-orientation)\\

% \section{Answers}
% \begin{itemize}
%     \item Polarizations and cell-interactions only set from biologically founded initial conditions.
%     \item No  timing
%     \item Stable under perturbations
%     \item With a p-value of 
% \end{itemize}


% \textbf{We are in a way the first who can simulate the posterior invagination (PMG) \pmg{A3} as this requires both successful modelling of \vf{A1} and \gb{A2}. Each of which has had multiple YYY before, but never in a way where combining them was possible}\\\\


% % \textbf{List of things that we know we have not taken into account but very likely could have been helpful}:
% % Cell shape change, Dorsal folds, Cephalic fold.

% % \textbf{Comparisons to data}\\
% % We have found through multiple comparisons, how our simulation is recapitulating the dynamics.

% % The movement vectors, timing, strain-distribution over time and rosettes all had clear relations[?] to reality while also showing the different shortcomings where our methodology could be improved.
% % \textbf{Further work}

% % We have identified the possibility of cell shape change (i.e. anisotropic relaxation distances in relation to external pressure) to be the most promising next step. Some work was already put into this witout anything fruitful. Other potential expansions on the model would be for the dorsal and cephalic folds to buckle under pressure. Also adding in the in-plane proliferation in the cephalic region. This was implemented, but removed for simplicity.

% % \textbf{Combinations}
% We have seen that every part of the simulated embryo was necessary for successful gastrulation. 
% While both our model and nature is stable to noise and YYY, the environment, boundary conditions and initial conditions are vital for defining YYY

% \textbf{closing remarks}



% \textit{axiomatic}dk