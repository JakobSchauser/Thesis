In 1952, two years before his untimely death, Alan Turing wrote an article called
\textit{The chemical basis of morphogenesis}.\cite{turing52the} In it, he predicted that off-equilibrium chemicals, with . He coined the term morphogens thereby cementing .\footnote{This paper was written \textit{one year} before the discovery of DNA and still use the word "gene" as its old "functional unit of heredity"} Much like cellular automata, 



It is suggested that a system of chemical substances, called morphogens, reacting together and diffusing through a tissue, is adequate to account for the main phenomena of morphogenesis through the emergence of genetic patterning.

When studying the interplay between morphogens and morphogenesis, no other organism has been studied as much as Drosophila Melongaster (the common fruit fly among friends).



\section{Why do we care about emergence?}
Leptin (1999) Review

From morphogen to morphogenesis and back
\section{Why do we care about Morphogenesis}
Plato used the words \textit{form} and \textit{idea} interchangeably

Morphogenesis involve coordination

All cells use the same array of actions to achieve wildly different results.  

"Morphogenesis in embryogenesis is one of the most interesting and rich sources of challenging problems in biomechanics."

As mentioned earlier, the  from the genetic makeup, local cell-neighborhood, global environment and maternal effects, all have an influence in a way YY many factors seems to corroborate successful embryogenesis. 

Combining the (1) limited set of actions (2) Timing/coordination (3)information sharing/retention (4) something else, makes the subject ripe for us(?):

Simulating these individual parts grants us the possibility for a potential bottoms-up understanding of the complete/whole.



\section{Why do we care about Modeling}
Main idea: Using a simple model, can complex, staggered morphological events emerge. And can such a simple model be used for making predictions?
Main idea2: How much is information is single-cell level and how much is BC, etc.