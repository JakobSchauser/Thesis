% IMPORTANT! In order for the document to compile, one needs to use XeLaTeX or LuaLaTeX as compiler. This can be done in  Overleaf by Menu -> Settings -> Compiler -> Choose XeLaTeX/LuaLaTeX

%% These are the packages and commands that I used, feel free to remove and edit them.

%% Set the path for your images. Remember to edit all your includegrpahics to the correct location.
 \graphicspath{ {images/} }

%% Suggested Commands %%

%% Personal favourite command if you want to number a single line in a series of equations. Use an align* environement and use \numberthis specifically on the line that you want to number rather than using an align environement and \notag on every line you don't want to number.  
\newcommand\numberthis{\addtocounter{equation}{1}\tag{\theequation}}

%% If you have Dutch names in your references 
\DeclareRobustCommand{\VAN}[3]{#2} % proper Dutch 'van/de' capitalisation

%% If you want your theorems to include the Chapter numbers in the nmumbering
\usepackage{amsthm} %P ackage that gives the theorems environment

\newtheorem{prop}{Proposition}[chapter]
\newtheorem*{prop*}{Proposition}
\newtheorem{example}{Example}[chapter]
\newtheorem{proof_sketch}{Sketch of the Proof}[chapter]
\newtheorem{theorem}{Theorem}[chapter]
\newtheorem{lemma}{Lemma}[chapter]
\newtheorem*{lemma*}{Lemma}
\newtheorem{remark}{Remark}[chapter]
\newtheorem{definition}{Definition}[chapter]
\newtheorem{corollary}{Corollary}[chapter]
\newtheorem{notation}{Notation}[chapter]


%% Other packages I typically use. 

% \usepackage{geometry}
% \usepackage{setspace} % Package for linespacing
% \usepackage{tabularx} % Package for table
% \usepackage{fontspec}

% \usepackage[T1]{fontenc}    % use 8-bit T1 fonts
% \usepackage{url}            % simple URL typesetting
% \usepackage{booktabs}       % professional-quality tables
% \usepackage{amsfonts}       % blackboard math symbols
% \usepackage{nicefrac}       % compact symbols for 1/2, etc.


% \usepackage{xcolor}         % colors
% \usepackage{dsfont}
% \usepackage{bm}
% \usepackage{mathtools}
% \usepackage{cleveref}
% \usepackage{amsmath}
% \usepackage{amssymb}
% \usepackage{xifthen}
% \usepackage[ruled]{algorithm2e}
% \usepackage{algorithmic}
