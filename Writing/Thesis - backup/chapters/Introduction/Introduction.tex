
\section{Questions}
Q0 - Mimimal umber of rules for vivo-like gastrulation to emerge\\
Q1 - Are BC/IC sufficient?\\
Q2 - Given a bottom-up, full scale model, what can we say re: domain-interactions\\
\\\\
Result 1: Visual and quantitative comparison between Silico and Vivo\\
Result 2a: Comparison between Silico mutants and Vivo mutants\\
Result 2a: Comparison between Silico mutants and Silico optimal (remember to include PCP-orientation)\\

This is mainly a phenomenological study in simulating the emergence of morphology.
\section{The rest}
In 1952, two years before his untimely death, Alan Turing wrote an article called
\textit{The chemical basis of morphogenesis}.\cite{turing52the} In it, he predicted that the diffusion of off-equilibrium chemicals, combined with relatively simple interactions could originate much of the extraordinary patterning we see in nature.\footnote{This paper was written \textit{one year} before the discovery of DNA and still only use the word "gene" in its vague "functional unit of heredity"-meaning} When talking about these chemicals, Turing coined the term \textit{morphogens}, hinting to the ancient Greek word for shape or form. 

As to YYY, in a paper Turing governing of physical shape, Turing that patterning in nature has a connection to not just visual pigmentations, but also the. \\


The YYY of is an example of emergence
\subsection{Why emergence. }

Understanding emergence is key to understanding many parts of physics. Just like the individual water molecule does not know the shape of a wave, no single cell knows what a human is. 

% but somewhere in its creation, the person must be imbued with something making it greater than the sum of its part. Some might call this emergent phenomenon a "soul", but you could also ascribe it to coordinated group behavior and smart feedback loops. \note{maybe too much? Maybe mention Hoefstadters strange loops? }

Most biological systems involve intricate inter-cell coordination. Most of these cells use the same array of seemingly simple actions to achieve wildly different and complex results. This coordination needs to robust and precise to make multicellular life feasible. In drug creation, the understanding of the interplay between the microscopic and the macroscopic is vital. Drugs often make their changes on minute intracellular dynamics -- with their desired effects on organ/human/even population scale.\footnote{The Danish population consist of $10^{20}$ cells}
Understanding the individual parts might grants us the possibility for an alternative, bottoms-up understanding of the whole. Understanding  


\textbf{Me ranting: Use some for abstract?}

Like snowflakes:
Proteins are made up of only 22 amino acids, their 

Nature keeps remaking the same structures. A leaf consists of recursive branching, where each branch is a smaller copy of the branch it is rooted in. But so are lungs and kidneys also made!(something something scales)  The symmetries from the building blocks define all life as we know it. 

Much like twins who are genetically identical can turn out differently, cells

Creation requires an interplay between information stored within the building blocks and their environment. (something something charcoal drawing)

The formation of biology cannot be separated from the chemistry of its constituents. And chemistry is just physics, 

\subsection{Why do we care about Morphogenesis?}
Plato used the words \textit{form} and \textit{idea} interchangeably. This is ascribed to a quirk of translation from ancient Greek, but he might have been more right than he realized.\\ Almost every biological structure has an intrinsic link between form and function; from the nephron in the kidney to the neck of the giraffe.  
On top of this, Nature is remarkably consistent. Any two livers across organisms of the same species are functionally the same. The translation from blueprint to finished structure must either be very precisely controlled or full of innate error correction. This makes the biological formation of physical shape (i.e. morphogenesis) an inherently interesting thing to study. 

\subsection{Why drosophila}
When studying the interplay between morphogens and morphogenesis, no other organism has been studied as much as the common fruit fly. Drosophila Melongaster, as it is known in the scientific community, is the definitive model organism and studying its biology has taught us many things that has been generalized to humans. \\


When studying any of the above-mentioned phenomena, physicsist and biologist seem too often to rely on the reductionist mathematics of solvable differential equations. Much like simple cellular automata can be used to describe things everything from the stock market to the Eucalyptus-leaves, we will be trying to apply many of the same principles to a 3d point-based model.

\subsection{Why modeling full-embryo gastrulation }
As questions!




% It is suggested that a system of chemical substances, called morphogens, reacting together and diffusing through a tissue, is adequate to account for the main phenomena of morphogenesis through the emergence of genetic patterning.




In this thesis we will be exploring \textbf{morphogenesis} through \textbf{modeling} the \textbf{emergent} behavior of individual cells during a the early stages of  gastrulation in \textbf{drosophila}.

We will be trying to find the minimal number of rules for vivo-like gastrulation to emerge. We will have extra focus on the speciality of the problem, only transmitting global information through the local sensing of boundary- and initial conditions.

To examine we will do both visual and quantitative comparisons between Silico and Vivo, not only for wild-type but also for different known (and maybe even some unknown) mutants. Putting our results under scrutiny 

Finally when we are confident in our model, we will take advantage of the\footnote{world first, I might add} bottom-up, full scale model to see what can we say regarding interactions between different domains.\\


% Also, as physicists, embryogenesis is a cornucopia of interesting examples of biomechanics. The genetic makeup, the cells local neighborhood and any global environment and maternal effects, all have an influence on successful embryogenesis. 


% \note{Maybe steal something from the following abstract:}\\
% \textit{From morphogen to morphogenesis and back}

% A long-term aim of the life sciences is to understand how organismal shape is encoded by the genome. An important challenge is to identify mechanistic links between the genes that control cell-fate decisions and the cellular machines that generate shape, therefore closing the gap between genotype and phenotype. The logic and mechanisms that integrate these different levels of shape control are beginning to be described, and recently discovered mechanisms of cross-talk and feedback are beginning to explain the remarkable robustness of organ assembly. The 'full-circle' understanding of morphogenesis that is emerging, besides solving a key puzzle in biology, provides a mechanistic framework for future approaches to tissue engineering.

