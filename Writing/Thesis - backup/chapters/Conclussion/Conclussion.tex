\section{Conclusion}

Q0 - Mimimal umber of rules for vivo-like gastrulation to emerge\\
Q1 - Are BC/IC sufficient?\\
Q2 - Given a bottom-up, full scale model, what can we say re: domain-interactions\\
\\\\
Result 1: Visual and quantitative comparison between Silico and Vivo\\
Result 2a: Comparison between Silico mutants and Vivo mutants\\
Result 2a: Comparison between Silico mutants and Silico optimal (remember to include PCP-orientation)\\


\begin{itemize}
    \item Polarizations and cell-interactions only set from biologically founded initial conditions.
    \item No explicit timing
    \item No chemotaxis
    \item No large-scale "information sharing"
    \item Stable under perturbations
    \item With a p-value of 
\end{itemize}

While looking at the [small parts] in many cases can lead to crudely reductionist conclusions, 

We have seen that even if many biological systems are "more than than the sum of its parts", having a model for the parts can allow for .

We have shown that complex time-[something] of events can arise from the physically simple interactions with boundary conditions and mechanical feedback-loops.

Turing showed that simple interplay between chemicals can create arbitrarily complex biopatterning that can serve as a blueprint. We have taken a step in the direction of showing how this basis can be utilized by many individual cells to the creation of physical form.  [something with information theory] 
\section{Discussion}
