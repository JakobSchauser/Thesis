All biological systems involve intricate inter-cell coordination. Most of these cells use the same array of seemingly simple actions to achieve wildly different and complex results. This coordination needs to robust and precise to make multicellular life feasible. \\

In this thesis we explore how emergent phenomena at multiple different length-scales help form the basis for morphogenesis. Focusing on fruit fly gastrulation, we propose a biologically founded, agent-based model with a minimal rule-set. Through both qualitative and quantitative analyses we examine the model's capabilities and shortcomings. We show that the only global information sharing needed for arriving at dynamics akin to what is observed in vivo can be contained in the initial conditions. Finally, we use the model to probe different known morphogenic events, seeing how their complex interplay are vital in shaping the animal to come.