\section{Questions}
\todo{Give me all the questions again}
Q0 - Mimimal umber of rules for vivo-like gastrulation to emerge\\
Q1 - Are BC/IC sufficient?\\
Q2 - Given a bottom-up, full scale model, what can we say re: domain-interactions\\
\\\\
Result 1: Visual and quantitative comparison between Silico and Vivo\\
Result 2a: Comparison between Silico mutants and Vivo mutants\\
Result 2a: Comparison between Silico mutants and Silico optimal (remember to include PCP-orientation)\\

\section{Answers}
\begin{itemize}
    \item Polarizations and cell-interactions only set from biologically founded initial conditions.
    \item No explicit timing
    \item No chemotaxis
    \item No large-scale "information sharing"
    \item Stable under perturbations
    \item With a p-value of 
\end{itemize}

While looking at the [small parts] in many cases can lead to crudely reductionist conclusions, 

We have seen that even if many biological systems are "more than than the sum of its parts", having a model for the parts can allow for .

We have shown that complex time-[something] of events can arise from the physically simple interactions with boundary conditions and mechanical feedback-loops.



\textbf{Comparisons to data}\\
We have found through multiple comparisons, how our simulation is recapitulating the dynamics.

The movement vectors, timing, strain-distribution over time and rosettes all had clear relations[?] to reality while also showing the different shortcomings where our methodology could be improved.
\textbf{Further work}

We have identified the possibility of cell shape change (i.e. anisotropic relaxation distances in relation to external pressure) to be the most promising next step. Some work was already put into this witout anything fruitful. Other potential expansions on the model would be for the dorsal and cephalic folds to buckle under pressure. Also adding in the in-plane proliferation in the cephalic region. This was implemented, but removed for simplicity.

\textbf{Combinations}
We have seen that every part of the simulated embryo was necessary for successful gastrulation. 
While both our model and nature is stable to noise and YYY, the enviroment, boundary conditions and initial conditions are vital for defining YYY

\textbf{closing remarks}
Turing showed that simple interplay between chemicals can create arbitrarily complex biopatterning that can serve as a blueprint. We have taken a step in the direction of showing how a combination between minimal environmental information and simple rules for the interplay between individual cells can lead to the creation of physical form. \todo{rephrase}
