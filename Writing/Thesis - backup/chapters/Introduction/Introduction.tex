
In 1952, two years before his untimely death, Alan Turing wrote an article called \textit{The chemical basis of morphogenesis}.\cite{turing52the} In it, he predicted that the diffusion of off-equilibrium chemicals, combined with relatively simple interactions could originate much of the extraordinary patterning we see in nature. When talking about these chemicals, Turing coined the term morphogens thereby cementing that patterning has a connection to not just visual pigmentation, but also physical shape. \\


The fact that simple interactions between individual constituents can lead to complex and surprising behavior on a larger scale, (i.e. emergence), is one of the core concepts in physics. But just like no individual water molecule knows the shape of a wave, no single cell knows what a human is. All biological systems involve intricate inter-cell coordination, which needs to be robust and precise for multicellular life to be feasible. Most of these cells use the same array of seemingly simple actions to achieve wildly different and complex results. On top of this, Nature is remarkably consistent. Any two livers across organisms of the same species are functionally the same. The translation from blueprint to finished structure must either be very precisely controlled or full of innate error correction. \\

Taking our onset in this mindset, we will in this thesis be introducing a model that, through simulating the emergent behavior of individual cells, can capture the full-embryo large scale movements that is seen in nature. We will be looking at morphogenesis, the formation of physical form, in early stages of the fruit fly embryo. The common fruit fly (Drosophila among friends) has been studied for decades and is used as a model for understanding many part of the earliest stages of life. Especially \textit{gastrulation}, the point in which the inside- and outside domains are defined and separated has been studied intensively and has birthed multiple Nobel prizes. We will be highlighting three events during the approximate 15 minutes this gastrulation takes, and through simulation of these, we will be corroborating the validity of our model. By perturbing the model and seeing how the events independently and collectively react we will be showcasing the novelty, justification and value of our bottom-up approach.\\

Plato used the words \textit{form} and \textit{idea} interchangeably. This can be ascribed to a quirk of translation from ancient Greek, but he might have been more correct than he realized: Every biological structure has an intrinsic link between form and function; from the beak of a bird to the neck of the giraffe. The fact that physical shape is so fundamental to nature raises questions about the processes that bring it into being.\\
How form arises, how cells coordinate, and how this can be modeled will be explored in the following chapters.













% The combination of a simple rule-set and initial condition can bring a total to be more than the sum of parts.
% Much like twins, who are genetically identical can turn out differently, cells that contain exact copies of a DNA string depend out outside factors to differentiate. \reph


% In this thesis we will be looking at how emergent phenomena across many different 


% \subsection{Why emergence. }


% but somewhere in its creation, the person must be imbued with something making it greater than the sum of its part. Some might call this emergent phenomenon a "soul", but you could also ascribe it to coordinated group behavior and smart feedback loops. \note{maybe too much? Maybe mention Hoefstadters strange loops? }

%  While we can 






% Understanding the individual parts might grants us the possibility for an alternative, bottoms-up understanding of the whole.


% \textbf{Me ranting: Use some for abstract?}

%  -- Plato -- ontological --
% Like snowflakes:
% Proteins are made up of only 22 amino acids, their 

% Nature keeps remaking the same structures. A leaf consists of recursive branching, where each branch is a smaller copy of the branch it is rooted in. But so are lungs and kidneys also made!(something something scales)  The symmetries from the building blocks define all life as we know it. 

% Much like twins who are genetically identical can turn out differently, cells

% Creation requires an interplay between information stored within the building blocks and their environment. (something something charcoal drawing)

% The formation of biology cannot be separated from the chemistry of its constituents. And chemistry is just physics, 

% \subsection{Why do we care about Morphogenesis?}


% \subsection{Why drosophila}
% When studying the interplay between morphogens and morphogenesis, no other organism has been studied as much as the common fruit fly. Drosophila Melongaster, as it is known in the scientific community, is the definitive model organism and studying its biology has taught us many things that has been generalized to humans. \\


% When studying any of the above-mentioned phenomena, physicsist and biologist seem too often to rely on the reductionist mathematics of solvable differential equations. Much like simple cellular automata can be used to describe things everything from the stock market to the Eucalyptus-leaves, we will be trying to apply many of the same principles to a 3d point-based model.

% \subsection{Why modeling full-embryo gastrulation }
% As questions!




% % It is suggested that a system of chemical substances, called morphogens, reacting together and diffusing through a tissue, is adequate to account for the main phenomena of morphogenesis through the emergence of genetic patterning.






% We will be trying to find the minimal number of rules for vivo-like gastrulation to emerge. We will have extra focus on the speciality of the problem, only transmitting global information through the local sensing of boundary- and initial conditions.

% To examine we will do both visual and quantitative comparisons between Silico and Vivo, not only for wild-type but also for different known (and maybe even some unknown) mutants. Putting our results under scrutiny 

% Finally when we are confident in our model, we will take advantage of the world first, bottom-up, full scale model to see what can we say regarding interactions between different domains.\\





% In this

% Also, as physicists, embryogenesis is a cornucopia of interesting examples of biomechanics. The genetic makeup, the cells local neighborhood and any global environment and maternal effects, all have an influence on successful embryogenesis. 


% \note{Maybe steal something from the following abstract:}\\
% \textit{From morphogen to morphogenesis and back}

% A long-term aim of the life sciences is to understand how organismal shape is encoded by the genome. An important challenge is to identify mechanistic links between the genes that control cell-fate decisions and the cellular machines that generate shape, therefore closing the gap between genotype and phenotype. The logic and mechanisms that integrate these different levels of shape control are beginning to be described, and recently discovered mechanisms of cross-talk and feedback are beginning to explain the remarkable robustness of organ assembly. The 'full-circle' understanding of morphogenesis that is emerging, besides solving a key puzzle in biology, provides a mechanistic framework for future approaches to tissue engineering.

