All biological systems involve intricate inter-cell coordination. Most of these cells use the same array of seemingly simple actions to achieve wildly different and complex results. This coordination needs to robust and precise to make multicellular life feasible. \\

In this thesis we explore how emergent phenomena at multiple scales help facilitate morphogenesis, the construction of physical shape. Focusing on fruit fly (Drosphila melongaster) gastrulation, we propose a biologically founded, agent-based model with a minimal rule-set. Using gene expression data, we reduce full-embryo gastrulation to a few cell-level parameters.
Through qualitative and quantitative analyses, the dynamics of the simulation are compared to what is observed in vivo, confirming overall alignment. 
Finally, we use the model to perturb known morphogenic events, observing how their complex interplay are vital in shaping the resulting animal.